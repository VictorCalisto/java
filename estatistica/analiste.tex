\documentclass[a4paper,12pt]{article}
\usepackage[utf8]{inputenc}
\usepackage{amsmath, amssymb}
\usepackage{geometry}
\geometry{margin=1in}
\title{Análise Matemática da Vantagem da Casa no Jogo de Caça-Níqueis}
\author{}
\date{}

\begin{document}
\maketitle

\section*{Cálculo da Probabilidade de Vitória}

Dado um tabuleiro de tamanho $3 \times 3$ com símbolos sorteados aleatoriamente e independentemente, onde há 13 símbolos possíveis (``A'', ``2'', ..., ``K''), a probabilidade de obter uma linha com três símbolos idênticos (sem considerar curingas) é:

\begin{equation}
P = \left( \frac{1}{13} \right) \cdot \left( \frac{1}{13} \right) \cdot \left( \frac{1}{13} \right) = \frac{1}{2197}
\end{equation}

Como existem 3 linhas horizontais, 3 verticais e 2 diagonais, temos 8 linhas no total que podem resultar em vitória. Assim, a probabilidade de obter \textbf{pelo menos uma linha vencedora} (com símbolos iguais) é estimada por:

\begin{equation}
P_{\text{total}} = 8 \cdot \frac{1}{2197} \approx 0{,}00364 \quad \text{ou} \quad 0{,}364\%
\end{equation}

\section*{Relação entre Probabilidade e Retorno (Odds)}

A \textbf{odds justa} para o jogador, baseada na probabilidade de vitória, seria o inverso da probabilidade:

\begin{equation}
\text{Odds} = \frac{1}{0{,}00364} \approx 274{,}625
\end{equation}

Ou seja, um pagamento justo baseado em risco seria de aproximadamente \textbf{274,625 vezes} o valor apostado.

Contudo, o sistema paga apenas:

\begin{equation}
\text{Pagamento Real} = 10 \times \text{aposta}
\end{equation}

\section*{Vantagem da Casa}

A \textbf{expectância matemática do jogador} ao apostar $R\$1{,}00$ é:

\begin{equation}
E = 0{,}00364 \cdot 10 + (1 - 0{,}00364) \cdot 0 = 0{,}0364
\end{equation}

Portanto, a \textbf{esperança de retorno} do jogador é de apenas \textbf{3,64 centavos por real apostado}, implicando em uma \textbf{perda esperada de R\$ 0{,}9636} a cada R\$1 apostado.

A \textbf{vantagem da casa}, portanto, é:

\begin{equation}
\text{House Edge} = 1 - E = 1 - 0{,}0364 = 0{,}9636 \quad \text{ou} \quad 96{,}36\%
\end{equation}

\section*{Conclusão}

A análise matemática demonstra que o jogo é altamente desfavorável ao jogador. A vantagem estatística da casa (house edge) de \textbf{96,36\%} significa que, em longo prazo, o sistema retém quase toda a quantia apostada, remunerando o jogador muito abaixo da odds justa baseada na probabilidade de vitória real. Este tipo de jogo é projetado para gerar lucro consistente para o operador, não para o apostador.
\end{document}
